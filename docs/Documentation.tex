\documentclass[12pt,             % Schriftgroesse
               a4paper,          % Papierformat
               %liststotoc,      % (alt) Tabellen- und Abbildungsverzeichnis im Inhaltsverz.
               listof=totoc,     % Tabellen- und Abbildungsverzeichnis im Inhaltsverz.
               %idxtotoc,        % (alt) Index im Inhaltsverz. auffuehren
               index=totoc,      % Index im Inhaltsverz. auffuehren
               %bibtotoc,        % (alt) Literaturverzeichnis im Inhaltsverz. auffuehren
               bibliography=totoc,% Literaturverzeichnis im Inhaltsverz. auffuehren
               oneside,         % auskommentieren, wenn beidseitig gedruckt wird
                                 % default ist 'twoside'
               BCOR1cm,          % zusaetzlicher Bindungsrand
               english   %  Englisch als weitere Sprache, Deutsch als Hauptsprache
               ]{scrbook}

\usepackage[utf8]{inputenc}  
\usepackage[T1]{fontenc}
\usepackage{lmodern}         
%\usepackage{eurosym}  
\usepackage[ngerman]{babel}
\usepackage{graphicx}
\usepackage[left=20mm, right=20mm, top=25mm]{geometry}
\usepackage{listings} \lstset{numbers=left, numberstyle=\tiny, numbersep=5pt, basicstyle =\footnotesize} 
\lstset{language=[sharp]C}
\usepackage{url}
\setlength\parindent{0pt} 
\usepackage{float}

\usepackage{hyperref}
\linespread{1.3}
\usepackage{multicol}
\usepackage[german]{varioref}

\bibliographystyle{alphadin}

\setcounter{tocdepth}{1}

\usepackage[]{acronym}

\title{Testdriven Development am Beispiel der ozb Webapplikation}
\author{Martin Briewig, Michael Leibel}


\begin{document}

\pagestyle{empty}
\begin{center}

\begin{figure}[h]
	\begin{center}
		\includegraphics[width=0.70\textwidth]{\string"Media/hskalogo\string".jpg}
	\end{center}
	\label{fig:hska_logo}
\end{figure} 

\vspace{5\baselineskip}

\textsc{\Huge Test-Driven Development\\ mit Ruby on Rails}\\[0.25cm]
\textsc{\large am Beispiel der o/ZB Webapplikation}\\[1.5cm]


\vspace{2\baselineskip}

\textsc{\LARGE Projektarbeit}\\[1.5cm]


\vspace{5\baselineskip}

vorgelegt von\\
Briewig, Martin, B.Sc. (Matr.-Nr.: 43509)\\
Leibel, Michael, B.Sc. (Matr.-Nr.: 43674)

\vspace{\baselineskip}

Betreuer: Prof. Dr. Frank Schaefer

Fachbereich: Informatik (Master)
\end{center}

\clearpage{}

\pagenumbering{roman} 

\tableofcontents{}

\clearpage{}

%\addcontentsline{toc}{section}{Abkürzungsverzeichnis}
\chapter*{Abkürzungsverzeichnis}

\begin{acronym}[SQL]
	\acro{AP}{Arbeitspaket}
	\acro{o/ZB}{Ohne Zins Bewegung}
\end{acronym}

\newpage

\pagenumbering{arabic} 
\chapter{Einleitung}
\section{Aufgabe}
% Tests
% Dokumente (Word,PDF) generieren

\clearpage
\chapter{Hintergrundwissen}
\section{Das o/ZB Projekt}

\clearpage
\section{Test-Driven Development}
\subsection{Definition}
\subsection{Unit-Tests}
\subsection{Functional-Tests}
\subsection{Integration-Tests}

\section{o/ZB Darlehensvertrag}
% Wie sieht der aus?

\clearpage
\chapter{Analyse}
\section{TDD}
% Ist-Analyse: Es gibt keine Tests, Fehlerbehaftete Anwendung
% Soll-Analyse: bugfrei durch TDD. Was müssen wir dafür testen?
% Auswertung der bisherigen manuellen Tests
% Implementierte Geschäftsprozesse verstehen (für die spätere Entwicklung der Tests)
% Vorgehensweise

\section{Dokumenten-Export}
% Was muss in den Darlehensvertrag? s.Hintergrundwissen
% Muss er noch im Nachhinein verändert werden? -> Word oder PDF?

\clearpage
\chapter{Implementierung}

\clearpage
\chapter{Ergebnis und Ausblick}
\section{Ergebnis}
\section{Ausblick}

\clearpage
\chapter{Anhang}

\newpage
%\addcontentsline{toc}{chapter}{Abbildungsverzeichnis}
\listoffigures

\newpage
%\addcontentsline{toc}{chapter}{Tabellenverzeichnis}
\listoftables

\newpage
%\addcontentsline{toc}{chapter}{Listingsverzeichnis}
\lstlistoflistings

\newpage
%\addcontentsline{toc}{chapter}{Literatur}
\bibliography{sources}

\end{document}

