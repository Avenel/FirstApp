\documentclass[12pt,             % Schriftgroesse
               a4paper,          % Papierformat
               %liststotoc,      % (alt) Tabellen- und Abbildungsverzeichnis im Inhaltsverz.
               listof=totoc,     % Tabellen- und Abbildungsverzeichnis im Inhaltsverz.
               %idxtotoc,        % (alt) Index im Inhaltsverz. auffuehren
               index=totoc,      % Index im Inhaltsverz. auffuehren
               %bibtotoc,        % (alt) Literaturverzeichnis im Inhaltsverz. auffuehren
               bibliography=totoc,% Literaturverzeichnis im Inhaltsverz. auffuehren
               oneside,         % auskommentieren, wenn beidseitig gedruckt wird
                                 % default ist 'twoside'
               BCOR1cm,          % zusaetzlicher Bindungsrand
               english   %  Englisch als weitere Sprache, Deutsch als Hauptsprache
               ]{scrbook}

\usepackage[utf8]{inputenc}  
\usepackage[T1]{fontenc}
\usepackage{lmodern}         
%\usepackage{eurosym}  
\usepackage[ngerman]{babel}
\usepackage{graphicx}
\usepackage[left=20mm, right=20mm, top=25mm]{geometry}
\usepackage{listings} \lstset{numbers=left, numberstyle=\tiny, numbersep=5pt, basicstyle =\footnotesize} 
\lstset{language=[sharp]C}
\usepackage{url}
\setlength\parindent{0pt} 
\usepackage{float}

\usepackage{hyperref}
\linespread{1.3}
\usepackage{multicol}
\usepackage[german]{varioref}

\bibliographystyle{alphadin}

\setcounter{tocdepth}{1}

\usepackage[]{acronym}

\title{Testdriven Development am Beispiel der ozb Webapplikation}
\author{Martin Briewig, Michael Leibel}


\begin{document}

\pagestyle{empty}
\begin{center}

\begin{figure}[h]
	\begin{center}
		\includegraphics[width=0.70\textwidth]{\string"Media/hskalogo\string".jpg}
	\end{center}
	\label{fig:hska_logo}
\end{figure} 

\vspace{5\baselineskip}

\textsc{\Huge Test-Driven Development\\ mit Ruby on Rails}\\[0.25cm]
\textsc{\large am Beispiel der o/ZB Webapplikation}\\[1.5cm]


\vspace{2\baselineskip}

\textsc{\LARGE Projektarbeit}\\[1.5cm]


\vspace{5\baselineskip}

vorgelegt von\\
Briewig, Martin, B.Sc. (Matr.-Nr.: 43509)\\
Leibel, Michael, B.Sc. (Matr.-Nr.: 43674)

\vspace{\baselineskip}

Betreuer: Prof. Dr. Frank Schaefer

Fachbereich: Informatik (Master)
\end{center}

\clearpage{}

\pagenumbering{roman} 

\tableofcontents{}

\clearpage{}

%\addcontentsline{toc}{section}{Abkürzungsverzeichnis}
\chapter*{Abkürzungsverzeichnis}

\begin{acronym}[SQL]
	\acro{AP}{Arbeitspaket}
	\acro{o/ZB}{Ohne Zins Bewegung}
\end{acronym}

\newpage

\pagenumbering{arabic} 
\chapter{Einleitung}
Die ohne Zins Bewegung Stuttgart arbeitet nun seit einigen Jahren mit der Hochschule Karlsruhe zusammen an einer Webanwendung. Diese Webanwendung soll die Geschäftsprozesse der Bewegung unterstützen und verwalten. Im Laufe der letzten Jahre ist aus der ursprünglichen Idee eine komplexe Webanwendung entstanden, die nun gründlich auf Fehler überprüft und gegebenfalls ausgebessert werden soll. 
Diese Aufgabe wird nun uns, Herr Briewig und Herr Leibel, anvertraut. Im Folgenden wird eine detaillierte Aufgabenstellung formuliert, an dieser sich diese Projektarbeit messen lassen wird.  

\section{Aufgabenbeschreibung}
Die Aufgaben dieser Projektarbeit sind recht vielfältig. Sie lassen sich in 3 Phasen einteilen. Darüber hinaus gibt es für jede Phase ein Zeitlimit von etwa 3 Wochen.

\subsection{Phase 1: Korrekturen, Dokumentieren}
\textbf {(Duedate: 02.05.2013)}\\
% Korrekturen am ER Diagramm/Relationenmodell
%    -    ER Diagramm soll der Implementierung entsprichen
%    -    Darstellung aller PKs und FKs für den nicht historisierten Fall
Die erste Phase wird von den Korrekturen am ER-Diagramm und dem dazugehörigen Relationsmodell dominiert. Das ER-Diagramm weißt noch ein paar Unstimmigkeiten auf, die im Laufe der Zeit entstanden sind. Diese gilt es zu bereinigen und die vorgenommenen Korrekturen in der Implementierung umzusetzen. Ziel ist ein ER-Diagramm, welches den nicht-historisierten Zustand darstellt. Dieses ER-Diagramm soll 1:1 in der Implementierung vorzufinden zu sein. Es soll vereinfacht dargestellt werden, lediglich die Primär- und Fremdschlüssel sollen den Entitäten als Zusatzinformationen dienen.\\

% Dokumentieren:
%    -    DB-Migration, wie geht das?
%    -    Deployment, wie funktioniert das?
Neben den Korrekturen am ER-Diagramm sind Dokumentationen der Vorgehens- und Funktionsweise des Datenbank Migrationstools notwendig. Hierfür sind zwei Batch-Skripte angefertigt worden, deren Nutzungs- und Funktionsweisen zu dokumentieren sind.\\

Im Laufe der Zeit sind die verschiedensten Techniken zur Bereitstellung und Versionierung der ozb Webanwendung verwendet worden. Es gilt nun, einen einheitlichen Vorgang zu definieren der für die nachfolgenden Projektgruppen verwendet werden soll. Basierend auf der anzufertigten Dokumentation, kann die Bereitstellung und Versionierung stetig verbessert werden.\\

% Allgemeine Korrekturen:
%    -    WebImport
Zusätzlich soll in dieser Phase ein kritischer Fehler bereinigt werden. Der WebImport, welcher für den Import der Kontobewegungen verantwortlich ist, arbeitet nicht richtig. Zur Zeit ist der Datei-Upload ist auf dem Testsystem nicht möglich. Darüber hinaus kommt es bei einer lokalen Testumgebung zu Laufzeitfehlern. Auch die im Webfrontend angezeigte Anzahl der importierten Datensätze ist nicht korrekt.

\subsection{Phase 2: Korrektur der Punkteberechnung, Word Export}
\textbf {(Duedate: 23.05.2013)}\\
% Allgemeine Korrekturen:
%    -    Korrektur der Punkteberechnung
% Feature:
%    -    Dokumente (Word,PDF) generieren

\subsection{Phase 3: Einrichtung Testdriven Development}
\textbf {(Duedate: 13.06.2013)}\\
% Tests
%    -    Beschreibung der Testumgebung mit Sublime 2(?) und RSpec
%    -    Testfälle entwickeln


\clearpage
\chapter{Hintergrundwissen}
\section{Das o/ZB Projekt}

\section{Das Testsystem}
% URL, Maschine, Versionen... 

\section{Test-Driven Development}
\subsection{Definition}
\subsection{Unit-Tests}
\subsection{Functional-Tests}
\subsection{Integration-Tests}

\section{o/ZB Darlehensvertrag}
% Wie sieht der aus?

\clearpage
\chapter{Allgemeine Dokumentationen}
\section{Das aktuelle ER-Diagramm}
% Wie sieht es aus?
% Was wurde angepasst?
% Beschreibung der Veränderungen am Code

\section{Das aktuelle Relationenmodell}
% Darstellung des ER-Diagramms in Tabellenform

\section{Deployment und Versionskontrolle}
% Was ist das? Capistrano, Git
% Wie sieht aktuelle Setup aus?
% Workflow

\section{Datenbank Migration}
% Welche Skript gibt es und was tun diese?
% Wie wird das auf dem Server ausgeführt?

\clearpage
\chapter{Korrekturen, Bugfixes}
% WebImport
\section{WebImport}

% Punkteberechnung
\section{Punkteberechnung}

\clearpage
\chapter{Testdriven Development}
\section{Analyse}
% Ist-Analyse: Es gibt keine Tests, Fehlerbehaftete Anwendung
% Soll-Analyse: bugfrei durch TDD. Was müssen wir dafür testen?
% Auswertung der bisherigen manuellen Tests
% Implementierte Geschäftsprozesse verstehen (für die spätere Entwicklung der Tests)
% Vorgehensweise
\section{Implementierung}
\subsection{RSpec}

\chapter{Neue Features}
\section{Dokumenten-Export}
\subsection{Analyse}
% Was muss in den Darlehensvertrag? s.Hintergrundwissen
% Muss er noch im Nachhinein verändert werden? -> Word oder PDF?
\subsection{Implementierung}
\clearpage


\chapter{Ergebnis und Ausblick}
\section{Ergebnis}
\section{Ausblick}

\clearpage
\chapter{Anhang}

\newpage
%\addcontentsline{toc}{chapter}{Abbildungsverzeichnis}
\listoffigures

\newpage
%\addcontentsline{toc}{chapter}{Tabellenverzeichnis}
\listoftables

\newpage
%\addcontentsline{toc}{chapter}{Listingsverzeichnis}
\lstlistoflistings

\newpage
%\addcontentsline{toc}{chapter}{Literatur}
\bibliography{sources}

\end{document}

